
\begin{TP}[Morphing]

Le \textbf{morphing} ou \textbf{morphage} est un des effets spéciaux applicables à un dessin. Il consiste à fabriquer une animation qui transforme de la façon la plus naturelle et la plus fluide possible un dessin initial vers un dessin final. 

\partie{Construction d'une image}
\begin{enumerate}
 \item Construisez un repère (chaque élève du groupe le fait sur son cahier).\\[0.5em] \label{MultDivRel_TP1}
Placez les points suivants dans le repère :\\[0.5em]
\begin{center}
\renewcommand*\tabularxcolumn[1]{>{\centering\arraybackslash}m{#1}}
\begin{ttableau}{0.8\linewidth}{4}
$A(0 ; 1)$ & $B(-4 ; 1)$ & $C(0 ; 5)$ & $D(0 ; -1)$ \\
$E(-3 ; -1)$ & $F(-2 ; -3)$ & $G(3 ; -3)$ & $H(4 ; -1)$ \\
$I(3 ; -1)$ & $J(3 ; 3)$ & $K(1 ; 2)$ & $L(3 ; 1)$ \\
 \end{ttableau}
\end{center}
\vspace{0.3cm}
Reliez à la règle les points dans l'ordre alphabétique de $A$ jusqu'à $L$ puis tracez le segment $[DI]$.
 \item Cette figure tient dans un carré. Construisez ce carré en rouge.
 \end{enumerate}
        
\partie{Transformation} \label{MultDivRel_TP2}
Pour cette partie, le travail peut être réparti entre les différents membres du groupe. Voici plusieurs transformations subies par les coordonnées des points :
\begin{itemize}
 \item On échange son abscisse et son ordonnée. On obtient $A_1$, $B_1$ \ldots ;
 \item On double son abscisse. On obtient $A_2$, $B_2$ \ldots ;
 \item On double son ordonnée. On obtient $A_3$, $B_3$ \ldots ;
 \item On double son abscisse et son ordonnée. On obtient $A_4$, $B_4$ \ldots ;
 \item On ajoute 4 à son abscisse et $-3$ à son ordonnée. On obtient $A_5$, $B_5$ \ldots.
 \end{itemize}
\begin{enumerate}
 \setcounter{enumi}{2}
 \item Pour chacune de ces transformations, indiquez les nouvelles coordonnées de chaque point puis construisez la figure dans un nouveau repère et enfin écrivez une phrase pour indiquer ce qu'est devenu le carré rouge.
 \end{enumerate}
        
\partie{Chacun sa figure}
\begin{enumerate}
 \setcounter{enumi}{3}
 \item Construisez la figure de votre choix dans un repère (15 points au maximum). Faites bien attention à ce que tous les points aient des coordonnées entières. À partir du dessin, remplissez un tableau de points comme à la question \ref{MultDivRel_TP1}.
 \item Donnez ce tableau à un autre groupe pour qu'il réalise la figure puis une transformation de votre choix parmi celles de la partie \ref{MultDivRel_TP2}.
 \end{enumerate}

\end{TP}

%%%%%%%%%%%%%%%%%%%%%%%%%%%%%%%%%%%%%%%%%%%%%%%%%%%%%%%%%%%%%%%%%%%%%%
%%%%%%%%%%%%%%%%%%%%%%%%%%%%%%%%%%%
%%%%%%%%%%%%%%%%%%%%%%%%%%%%%%%%%%%
%MiseEnPage
%%%%%%%%%%%%%%%%%%%%%%%%%%%%%%%%%%%
\newpage
%%%%%%%%%%%%%%%%%%%%%%%%%%%%%%%%%%%
%%%%%%%%%%%%%%%%%%%%%%%%%%%%%%%%%%%



\begin{TP}[Le bon produit]

\partie{La construction du jeu}
\begin{enumerate}
 \item Avec du papier épais ou du carton, fabriquez  66 cartes à jouer.
 \item Au stylo bleu, fabriquez les 38 cartes « facteur » :
 \begin{itemize}
  \item Deux portent le nombre 0 ;
  \item Trois exemplaires pour chacun des nombres : $-9$ ; $-6$ ; $-4$ ; $-3$ ; $-2$ ; $-1$ ; 1 ; 2 ; 3 ; 4 ; 6 et 9.
  \end{itemize}
\underline{Remarque} : Soulignez les 6 et les 9 pour éviter de les confondre.
 \item Au stylo rouge, fabriquez les 28 cartes « produit » :
 \begin{itemize} 
  \item Deux portent le nombre 0 ;
  \item Les autres sont toutes différentes et portent les nombres : $-54$ ; $-36$ ; $-27$ ; $-24$ ; $-18$ ; $-16$ ; $-12$ ; $-9$ ; $-8$ ; 6 ; $-4$ ; $-3$ ; $-2$ ; 2 ; 3 ; 4 ; 6 ; 8 ; 9 ; 12 ; 16 ; 18 ; 24 ; 27 ; 36 et 54.
  \end{itemize}
 \end{enumerate}

\partie{Les règles du jeu} 
Chaque joueur reçoit six cartes « facteur » puis pioche une carte « produit ». Celui qui a le plus grand nombre joue en premier (en cas d'égalité, les joueurs ex-aequo piochent une deuxième carte produit). On tourne ensuite dans le sens des aiguilles d'une montre.\\[0.5em]
Les cartes « produit » piochées sont posées face visible. On complète de façon à en avoir 10 en tout sur la table.\\[0.5em]
Le joueur dont c'est le tour pioche une carte « produit » et la pose sur la table avec les autres.\\[0.5em]
Si, avec deux de ses cartes facteurs, il peut obtenir un des produits visibles, il écarte les trois cartes (les deux cartes « facteur » et la carte « produit »).\\[0.5em]
S'il ne peut pas, il pioche deux cartes « facteur » et regarde à nouveau s'il peut obtenir un produit.\\[0.5em]
S'il propose une combinaison et qu'il a fait une erreur de calcul, il pioche également deux cartes « facteur ».\\[0.5em]
C'est alors au tour du joueur suivant.\\[0.5em]
Lorsqu'un joueur a écarté toutes ses cartes « facteur », il a gagné.

\end{TP}

%%%%%%%%%%%%%%%%%%%%%%%%%%%%%%%%%%%
%%%%%%%%%%%%%%%%%%%%%%%%%%%%%%%%%%%
%MiseEnPage
%%%%%%%%%%%%%%%%%%%%%%%%%%%%%%%%%%%
\vfill
%%%%%%%%%%%%%%%%%%%%%%%%%%%%%%%%%%%
%%%%%%%%%%%%%%%%%%%%%%%%%%%%%%%%%%%

