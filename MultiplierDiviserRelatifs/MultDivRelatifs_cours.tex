%\section{Une section}

% remarque : pour qu'un mot se retrouve dans le lexique : \MotDefinition{asymptote horizontale}{} 

\begin{aconnaitre}
Pour multiplier deux nombres relatifs, on multiplie les valeurs absolues et on applique la \MotDefinition{règle des signes}{} :
\begin{itemize}
 \item Le produit de deux nombres relatifs de \textbf{même signe} est \textbf{positif} ;
 \item Le produit de deux nombres relatifs de \textbf{signes opposés} est \textbf{négatif}.
 \end{itemize}
\end{aconnaitre}

\begin{methode*1}[Multiplier deux nombres relatifs]

 \begin{exemple*1}
Effectue la multiplication : $E = (-4) \times (-2,5)$. \\[0.5em]
Le résultat est positif car c'est le produit de deux nombres négatifs :

$E = 4 \times 2,5$, \hspace{3cm} $E = 10$.
 \end{exemple*1}

 \begin{exemple*1}
Effectue la multiplication : $F = 0,2 \times (-14)$. \\[0.5em]
Le résultat est négatif car c'est le produit d'un nombre positif par un nombre négatif :

$F = -(0,2 \times 14)$, \hspace{3cm}

$F = -2,8$.
 \end{exemple*1}

 \exercice  
Effectue les multiplications suivantes :
\vspace{1em}
\begin{colenumerate}{3}
 \item $(-7) \times (-8)$ \dotfill;
 \vspace{1em}
 \item $-5 \times (-11)$ \dotfill;
 \vspace{1em}
 \item $(-9) \times 6$ \dotfill;
 \vspace{1em}
 \item $-8 \times 0,5$ \dotfill;
 \vspace{1em}
 \item $10 \times (-0,8)$ \dotfill;
 \vspace{1em}
 \item $(-7) \times 0$\dotfill.
 \end{colenumerate}
%\correction



 \end{methode*1}
 
 %%%%%%%%%%%%%%%%%%%%%%%%%%%%%%%%%%%%%%%%%%%%%%%%%%%%%%%%%%%%%%%%%%%%%%%%
 %%%%%%%%%%%%%%%%%%%%%%%%%%%%%%%%%%%
%%%%%%%%%%%%%%%%%%%%%%%%%%%%%%%%%%%
%MiseEnPage
%%%%%%%%%%%%%%%%%%%%%%%%%%%%%%%%%%%
\vspace{2cm}
%%%%%%%%%%%%%%%%%%%%%%%%%%%%%%%%%%%
%%%%%%%%%%%%%%%%%%%%%%%%%%%%%%%%%%%
 
 
 \begin{aconnaitre}
 \begin{itemize}
  \item Le produit de plusieurs nombres relatifs est \textbf{positif} s'il comporte un nombre \textbf{pair} de \textbf{facteurs négatifs} ;
  \item Le produit de plusieurs nombres relatifs est \textbf{négatif} s'il comporte un nombre \textbf{impair} de \textbf{facteurs négatifs}.
  \end{itemize}
\end{aconnaitre}

%%%%%%%%%%%%%%%%%%%%%%%%%%%%%%%%%%%
%%%%%%%%%%%%%%%%%%%%%%%%%%%%%%%%%%%
%MiseEnPage
%%%%%%%%%%%%%%%%%%%%%%%%%%%%%%%%%%%
\newpage
%%%%%%%%%%%%%%%%%%%%%%%%%%%%%%%%%%%
%%%%%%%%%%%%%%%%%%%%%%%%%%%%%%%%%%%

\begin{methode*1}[Multiplier plusieurs nombres relatifs]

 \begin{exemple*1}
Quel est le signe du produit : $A = -6 \times 7 \times (-8) \times (-9)$ ? \\[0.5em]
Le produit a trois facteurs négatifs. 3 est impair donc $A$ est négatif.
 \end{exemple*1}
 
  \begin{exemple*1}
Calcule le produit : $B = 2 \times (-4) \times (-5) \times (-2,5) \times (-0,8)$. \\[0.5em]
Il y a quatre facteurs négatifs; 4 est pair donc $B$ est positif : \\
$B = 2 \times 4 \times 5 \times 2,5 \times 0,8$ ; \hspace{1cm}  $B = (2 \times 5) \times (4 \times 2,5) \times 0,8$ ; \\       
$B = 10 \times 10 \times 0,8$ ; \hspace{1cm} $B = 80$.
 \end{exemple*1}
 
 \exercice  
Quel est le signe du produit $C = 9 \times (-9) \times (-9) \times 9 \times (-9) \times (-9) \times (-9)$ ?
%\correction
     
 \exercice  
Calcule :
\vspace{.5em}
\begin{colenumerate}{2}
 \item $-25 \times (-9) \times (-4)$ \dotfill;
 \item $0,5 \times 6 \times (-20) \times 8$ \dotfill.
 \end{colenumerate}
 %\correction

 \end{methode*1}
 
 %%%%%%%%%%%%%%%%%%%%%%%%%%%%%%%%%%%%%%%%%%%%%%%%%%%%%%%%%%%%%%%%%%%%%%%%
 
 \begin{aconnaitre}
Pour diviser deux nombres relatifs non nuls, on divise les valeurs absolues et on applique la \MotDefinition{règle des signes}{} :
\begin{itemize}
 \item Le quotient de deux nombres relatifs de \textbf{même signe} est \textbf{positif} ;
 \item Le quotient de deux nombres relatifs de \textbf{signes opposés} est \textbf{négatif}.
 \end{itemize}
\end{aconnaitre}

\begin{methode*1}[Diviser deux nombres relatifs]

 \begin{exemple*1}
Effectue la division suivante : $A = 65 : (-5)$. \\[0.5em]
Le résultat est négatif car c'est le quotient de deux nombres de signes opposés :

$65 : 5 = 13$ donc $A = -13$.
 \end{exemple*1}
 
 
 \begin{exemple*1}
Effectue la division: $B = (-30) : (-4)$. \\[0.5em]
Le résultat est positif car c'est le quotient de deux nombres négatifs :

$B = 30 : 4$,

$B = 7,5$.
 \end{exemple*1}
 
 \exercice  
Quel est le signe des quotients suivants ?
\vspace{1em}
\begin{colenumerate}{4}
 \item $56 : (-74)$ \dotfill;
 \item $(-6) : (-5)$ \dotfill;
 \item $9 : (-13)$ \dotfill;
 \item $-7 : (-45)$\dotfill.
 \end{colenumerate}
%\correction

 \exercice  
Calcule de tête :
\vspace{1em}
\begin{colenumerate}{4}
 \item $45 : (-5)$\dotfill;
 \item $(-56) : (- 8)$ \dotfill;
 \item $-59 : (-10)$ \dotfill;
 \item $-14 : 4$\dotfill.
 \end{colenumerate}
%\correction

 \end{methode*1}
 
 %%%%%%%%%%%%%%%%%%%%%%%%%%%%%%%%%%%%%%%%%%%%%%%%%%%%%%%%%%%%%%%%%%%%%%%%
