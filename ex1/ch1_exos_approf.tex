Pour les exercices \RefExercice{continuite_first} à \RefExercice{continuite_last}, on donne ci-dessous la\linebreak définition de continuité en un réel.

\begin{cadre}[A1][A4]
Soit $f$ une fonction définie sur un intervalle $I$ de $\mathbb{R}$ et $x_0\in I$.
$f$ est \textbf{continue en} $\boldsymbol{x_0}$ si $\displaystyle\lim_{x\rightarrow x_0}f(x)=f(x_0)$.
 \end{cadre}



% 41
  \begin{exercice}\label{continuite_first}
La fonction $f$ définie sur $\mathbb{R}$ par :
    \[f (x)= \left\{\begin{array}{@{}cl}
    \dfrac{2 - \sqrt{x + 3}}{x - 1} & \text{si~} x\neq1\\
    -\dfrac{1}{4} & \text{si~} x=1
    \end{array}\right. \]
est-elle continue en $1$ ?
\end{exercice}

\begin{exercice}
 La fonction $f$ définie sur $[-1~;~+\infty[$ par :
    \[f(x)=\left\{
      \begin{array}{ll}
       \dfrac{x + 1}{\sqrt{x + 1}} & \text{si } x>-1\\
       1 & \text{si } x=-1.
      \end{array}\right.\]
est-elle continue en $-1$ ?
\end{exercice}
%
% 10
%
\begin{exercice}
 Soit $k$ un entier et $f$ une fonction définie sur $\mathbb{R}$.\\
 Déterminer $k$ pour que $f$  soit continue sur $\mathbb{R}$.
\begin{enumerate}
\item $f(x) =\left\{\begin{array}{ll}
      x^2 - 5 & \text{si } x < 1 \\
       k & \text{si } x \geqslant 1
    \end{array}\right.$.
\item $f(x) =\left\{\begin{array}{ll}
    k & \text{si } x= -1 \\
    \dfrac{2x + \sqrt{x + 5}}{x + 1} & \text{si } x>-1
  \end{array}\right.$.
  \end{enumerate}
\end{exercice}


  \begin{exercice}\label{continuite_last}
   Soit $a$ un réel et $g$ la fonction définie sur
        $\mathbb{R}$ par :
      \[g(x)=\left\{\begin{array}{@{}ll}
        x^2+1    & \text{si~} x \leqslant 1 \\
        x^2+ax+a & \text{si~} x > 1
        \end{array}\right..\]
      Peut-on déterminer $a$ pour que $g$ soit continue sur $\mathbb{R}$ ?
\begin{corrige}
Corrigé d'un exercice de la partie approfondissement.
\end{corrige}
  \end{exercice}
  
  
  \begin{exercice}[« La science est l'asymptote de la vérité »\footnote{« La science est l’asymptote de la vérité. Elle approche sans cesse et ne touche jamais. » d'après Hugo, Victor, \emph{William Shakspeare}.}]
Rudy a remarqué qu'« \emph{une asymptote, c'est comme une tangente à l'infini} ».
Son professeur  digresse alors.
\begin{enumerate}
\item Soit $f$ la fonction homographique propre :
\[f(x)=\dfrac{ax+b}{cx+d}\]
définie sur $\mathcal{D}=\mathbb{R}\setminus\left\{-\dfrac{d}{c}\right\}$ avec $c\neq0$ et $ad-bc\neq0$.\par
« \emph{Monsieur, pourquoi "homographique \textbf{propre}" ?} ».\par
De quel type serait la fonction $f$ :
\begin{colitemize}{2}
\item pour $c=0$ ? \item pour $ad-bc=0$ ?
\end{colitemize}
\item Montrez que :
\begin{enumerate}
\item $f(x)=\dfrac{a}{c}-\dfrac{ad-bc}{c(cx+d)}$ pour $x\in\mathcal{D}$. \label{homo2a}
\item $f(x)=\left(\dfrac{a+bx^{-1}}{c+dx^{-1}}\right)$ pour $x\in\mathcal{D}^*$. \label{homo2b}
\item $f'(x)=\dfrac{ad-bc}{\left(cx+d\right)^2}$ pour $x\in\mathcal{D}$.
\end{enumerate}
\item Déduisez de \RefItem{homo2a} et \RefItem{homo2b} les équations des asymptotes à la courbe représentative de $f$ aux bornes de $\mathcal{D}$. \label{homo3}
\item Calculez les limites suivantes :\label{homo4}
\begin{colenumerate}{2}
\item $\displaystyle\lim_{x\to\pm\infty}f'(x)$  \item $\displaystyle\lim_{x\to -d/c}f'(x)$
\end{colenumerate}
« \emph{Plus ou moins l'infini, vous n'en êtes pas sûr ?} ».\par
Le professeur précise qu'il veut les limites de $f'(x)$ en $+\infty$ et $-\infty$.
\item Rapprochez les résultats du  \RefItem{homo4} de celui du \RefItem{homo3}.\par
Concluez à propos de la remarque de Rudy.
\end{enumerate}

\begin{corrige}
Corrigé d'un autre exercice de la partie approfondissement !!
\end{corrige}
\end{exercice}


