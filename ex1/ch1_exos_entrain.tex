 \definecolor{fondTI}{HTML}{869286}


\serie{Limites : interprétation graphique}

% cet exercice possède à droite du titre un renvoi vers une fiche méthode du cours (elle doit exister!!)
% la commande ExerciceRefMethode contient la référence de la boite méthode
% la commande label contient la référence de l'exercice
% par convention on met le même nom avec "exo" en plus pour la référence de l'exercice
\begin{exercice*}[~\hfill\ExerciceRefMethode{test_boite_methode}]\label{exo_test_boite_methode}
Exercice avec renvoi à une fiche méthode et corrigé.
Soit la fonction $f$ définie sur $\mathbb{R}$  par :
\[f(x)= x^2\left(1 - \dfrac{x^2}{9}\right).\]
  \begin{enumerate}
  \item Conjecturer les limites de $f$ en $+\infty$ et en
    $-\infty$ à partir de la représentation graphique ci-dessous obtenue à l'aide d'un logiciel.
  \item Étudier les limites de $f$ en $+\infty$ et en $-\infty$.
  \item Expliquer pourquoi la conjecture était erronée.
  \end{enumerate}
 \begin{corrige}
  Ici on range le corrigé de l'exercice
  
  Test sur le corrigé
\end{corrige}
\end{exercice*}


% cet exercice contient un logo "INFO" à droite du titre
\begin{exercice}[~\hfill\tice]
  Soit $g$ la fonction définie par :
  \[g(x) = \dfrac{x}{\sqrt{3x^2 + x + 7}}\] représentée par
  $\mathscr{C}$ dans un repère.
  \begin{enumerate}
  \item Donner l'ensemble de définition de la fonction $g$.
  \item À l'aide d'un logiciel de géométrie dynamique :
    \begin{enumerate}
    \item Tracer la courbe $\mathscr{C}$.
    \item Conjecturer une valeur approchée de la limite en $+\infty$
      de la fonction $g$.
    \end{enumerate}
    \item Déterminer par calcul la valeur exacte de
      la limite de $g$ en $+\infty$.
  \end{enumerate}
\end{exercice}

% un exercice simple

\begin{exercice}
Soit $f$ la fonction définie sur $\mathbb{R}\setminus\{-3~;~3\}$ par :
\[f(x)=\dfrac{1-3x}{x^2-9}.\]
\begin{enumerate}
\item Déterminer la limite de $f$ en $-\infty$ et $+\infty$.
\begin{enumerate}
\item Sur une calculatrice, on a tracé le graphe de $f$ ce qui a donné l'écran suivant :
\item Expliquer pourquoi il semble apparaître une contradiction.
\end{enumerate}
\end{enumerate}
\begin{corrige}
  Et hop, encore un autre corrigé !
\end{corrige}
\end{exercice}

\serie{Limites : opérations}

% un exercice avec un titre en plus
% en plus cet exercice propose des questions sur deux colonnes "colenumerate"
\begin{exercice}[En $\boldsymbol{-2}$, c'est rationnel !]
 Étudier la limite de la fonction $f$ en  $-2$.
   \begin{colenumerate}{2}
    \item $f(x)=\dfrac{x-4}{x^2+3x+2}$
    \item $f(x)=\dfrac{-x^2+x+6}{2x^2+5x+2}$
    \item $f(x)=\dfrac{x^2-4}{\left(x+2\right)^2}$
    \item $f(x)=\dfrac{x^3+8}{x^2-x-6}$
    \end{colenumerate}
\end{exercice}


\begin{exercice}[En $\boldsymbol{0}$, c'est radical !]
 Étudier la limite de la fonction $f$ en  $0$.
   \begin{colenumerate}{2}
    \item $f(x)=\dfrac{\sqrt{x+1}}{x}$
    \item $f(x)=\dfrac{\sqrt{x+1}-1}{x}$
    \item $f(x)=\dfrac{\sqrt{x+4}-2}{x}$
    \item $f(x)=\dfrac{\sqrt{1-x}-1}{x^2-2x}$
    \end{colenumerate}
\end{exercice}






\begin{exercice}
Déterminer les limites suivantes.
\begin{colenumerate}{2}
\item $\displaystyle \lim_{x\to+\infty} \dfrac{2x+3}{3x-2}$
\item $\displaystyle \lim_{\substack{x\to 1\\ x<1}} \dfrac{x-1}{x^2+x-2}$
\item $\displaystyle \lim_{x\to-\infty}\sqrt{\dfrac{2x-1}{x-2}}$
\item $\displaystyle \lim_{\substack{x\to 1\\ x>1}} \dfrac{x-1}{x^2+x-2}$
\end{colenumerate}
\end{exercice}

\begin{exercice}
Déterminer les limites suivantes.
\begin{colenumerate}{2}
\item $\displaystyle \lim_{x\to+\infty} \sqrt{5-\dfrac{4}{x^2}}$
\item $\displaystyle \lim_{x\to-\infty} \left(2-\dfrac{1}{x}\right)^3$
\item $\displaystyle \lim_{x\to+\infty} \left(x-\sqrt{x}\right)$
\item $\displaystyle \lim_{\substack{x\to 0\\ x>0}} \sqrt{\dfrac{2-x}{x}}$
\end{colenumerate}
\end{exercice}


\serie{Limites : comparaison/encadrement}


% un exercice avec qcm 

\begin{exercice}
Soit une fonction $f$ telle que $f(x)$ vérifie une \mbox{inégalité} ou un encadrement sur un ensemble donné.\\
Indiquer les limites qu'on peut en déduire parmi les deux proposées.
\begin{enumerate}
\item Pour tout réel $x\neq0$, on a $\dfrac{1}{x}\leqslant f(x)$.
    \begin{ChoixQCM}{2}
\item $\displaystyle\lim_{\substack{x\to 0\\ x<0}}f(x)$
\item $\displaystyle\lim_{\substack{x\to 0\\ x>0}}f(x)$
\end{ChoixQCM}
\item Pour tout réel $x\neq0$, on a $f(x)\leqslant \dfrac{1}{x}$.
    \begin{ChoixQCM}{2}
\item $\displaystyle\lim_{\substack{x\to 0\\ x<0}}f(x)$
\item $\displaystyle\lim_{\substack{x\to 0\\ x>0}}f(x)$
\end{ChoixQCM}
\item Pour tout réel $x>1$, on a $x+\dfrac{1}{x}\leqslant f(x)\leqslant x+1$.
    \begin{ChoixQCM}{2}
\item $\displaystyle\lim_{\substack{x\to 1\\ x>1}}f(x)$
\item $\displaystyle\lim_{x\to+\infty}f(x)$
\end{ChoixQCM}
\item Pour tout réel $x>0$, on a $-\dfrac{1}{x}\leqslant f(x)\leqslant \dfrac{1}{x}$.
    \begin{ChoixQCM}{2}
\item $\displaystyle\lim_{\substack{x\to 0\\ x>0}}f(x)$
\item $\displaystyle\lim_{x\to+\infty}f(x)$
\end{ChoixQCM}
\item Pour tout réel $x\in]0~;~1[$, on a $|f(x)-1|\leqslant x$.
    \begin{ChoixQCM}{2}
\item $\displaystyle\lim_{\substack{x\to 0\\ x>0}}f(x)$
\item $\displaystyle\lim_{\substack{x\to 1\\ x<1}}f(x)$
\end{ChoixQCM}
\end{enumerate}
\end{exercice}

