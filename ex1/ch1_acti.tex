\begin{activite}[Notion de limite]

Une première activité !

$2+2=4$

Sur la \textbf{demi-droite graduée} ci-dessous, quel est le nombre associé au point B ? Qu'est-ce qui te permet de l'affirmer ?

Ce nombre est associé à un événement historique important. Lequel ?
Décalque cette demi-droite et place le point N associé au nombre qui correspond à l'année de la chute du mur de Berlin.
Le nombre associé à un point sur une demi-droite graduée est l'\textbf{abscisse} de ce point.


\begin{partie}[une partie de l'activité...]
\vspace{-1.5em}
\begin{enumerate}
 \item Calculer $f(x)$ pour $x = 10$ ; 100 ; 1~000 ; $10^{4}$ ; $10^{5}$ ; etc.
 \item Que peut-on conjecturer quant à $f(x)$ lorsque $x \to +\infty$ ?
 \end{enumerate}
\end{partie}

\begin{partie}[... et une autre partie]
On vient de remarquer la propriété suivante, que l'on va par la suite chercher à démontrer (ah bon).
\end{partie}
\vspace{-4em}
\end{activite}



%%%%%%%%%%%%%%%%%%%%%%%%%%%%%%%%%%%%%%%%%%%%%%%%%%%%%%%%%%%%%%%%%%%%%%%%


\begin{activite}[Une autre activité]

\begin{partie}[partie 1]
Aux XVII\up{e} et XVIII\up{e} siècles, la notion de fonction. 
\end{partie}

\begin{partie}[Partie 2]
Au début du XIX\up{e} siècle, Bolzano et Cauchy.\end{partie}
\vspace{-4em}
\end{activite}


%%%%%%%%%%%%%%%%%%%%%%%%%%%%%%%%%%%%%%%%%%%%%%%%%%%%%%%%%%%%%%%%%%%%%%%%
%%%%%%%%%%%%%%%%%%%%%%%%%%%%%%%%%%%%%%%%%%%%%%%%%%%%%%%%%%%%%%%%%%%%%%%%
\pagebreak
\vspace*{-1cm}

\begin{activite}[Encore une activité (avec saut de page)]

\begin{partie}[Première partie]
On considère les deux fonctions $u$ et $v$ suivantes
\end{partie}

\begin{partie}[...Seconde partie]
Si $u$ et $v$ sont deux fonctions.

On donne les fonctions de référence $a$, $b$, $c$ et $d$ définies par :
\end{partie}

\begin{partie}[Partie 3]
Rien dans cette partie :)
\end{partie}
\end{activite}

\begin{debat}
Et là on peut mettre un petit débat.
\end{debat}

