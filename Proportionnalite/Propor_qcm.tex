
\QCMautoevaluation{Pour chaque question, plusieurs réponses sont
  proposées.  Déterminer celles qui sont correctes.}

\begin{QCM}
  \begin{GroupeQCM}
    \begin{exercice}
     1 CD coûte 6,50 CHF. Combien coûtent 11 CD ?
      \begin{ChoixQCM}{4}
      \item 65 CHF
      \item 71,5 CHF
      \item 715 CHF
      \item 11 CHF
      \end{ChoixQCM}
\begin{corrige}
     \reponseQCM{b}
   \end{corrige}
    \end{exercice}
    
   
    \begin{exercice}
      1 kg de pommes coûte 1,60 CHF. Rémi paye 1,20 CHF. Il a donc acheté \ldots
      \begin{ChoixQCM}{4}
      \item 750 g de pommes
      \item 0,40 kg de pommes
      \item 1,333 kg de pommes
      \item 0,75 kg de pommes
      \end{ChoixQCM}
\begin{corrige}
     \reponseQCM{ad}
   \end{corrige}
    \end{exercice}
    
    
    \begin{exercice}
      Quelle(s) est (sont) la (les) situation(s) de proportionnalité ?
      \begin{ChoixQCM}{4}
      \item Les dimensions d'une maquette par rapport aux dimensions de l'objet réel.
      \item La taille d'un être humain avec son âge.
      \item La quantité de peinture en fonction de la surface à peindre.
      \item Le prix à payer en fonction du nombre d'articles achetés.
      \end{ChoixQCM}
\begin{corrige}
     \reponseQCM{ac}
   \end{corrige}
    \end{exercice}
    
    
    \begin{exercice}
      Quel(s) est (sont) le (les) tableau(x) de proportionnalité ?
      \begin{ChoixQCM}{4}
      \item
       \begin{tabular}{|c|c|c|}
       \hline
        1 & 2 & 3 \\\hline
        4,5 & 9 & 13,5 \\\hline
        \end{tabular}
      \item
       \begin{tabular}{|c|c|c|}
       \hline
        1 & 2 & 6 \\\hline
        7 & 14 & 41 \\\hline
        \end{tabular}
      \item
       \begin{tabular}{|c|c|c|}
       \hline
        3 & 6 & 9 \\\hline
        7,5 & 15 & 21,5 \\\hline
        \end{tabular}
      \item
       \begin{tabular}{|c|c|c|}
       \hline
        5 & 10 & 20 \\\hline
        9 & 14 & 24 \\\hline
        \end{tabular}
      \end{ChoixQCM}
\begin{corrige}
     \reponseQCM{a}
   \end{corrige}
    \end{exercice}
    
    
    \begin{exercice}
      Si trois baguettes coûtent 2,40 CHF, alors \ldots
      \begin{ChoixQCM}{4}
      \item Cinq baguettes coûtent 4,40 CHF
      \item Dix baguettes coûtent 8 CHF
      \item Six baguettes coûtent 8,20 CHF
      \item Deux baguettes coûtent 1,60 CHF
      \end{ChoixQCM}
\begin{corrige}
     \reponseQCM{bd}
   \end{corrige}
    \end{exercice}
    
    
    \begin{exercice}
      8 fourmis de même taille, en file indienne, mesurent au total 7,2 cm, donc \ldots
      \begin{ChoixQCM}{4}
      \item 7 fourmis   mesurent au total 6,3 cm
      \item 12 fourmis   mesurent au total 10,2 cm
      \item 16 fourmis mesurent au total 144 mm
      \item 2 fourmis mesurent au total 1,6 cm
      \end{ChoixQCM}
\begin{corrige}
     \reponseQCM{ac}
   \end{corrige}
    \end{exercice}
    
    
     \begin{exercice}
      Un nénuphar double de surface tous les jours. En quarante jours, il recouvre un lac.
      \begin{ChoixQCM}{4}
      \item Le lac était recouvert à moitié le vingtième jour
      \item Le quatre‑vingtième jour le nénuphar couvrira deux lacs de même surface
      \item Un quart du lac était recouvert le trente‑huitième jour
      \item La situation présentée est proportionnelle
      \end{ChoixQCM}
\begin{corrige}
     \reponseQCM{c}
   \end{corrige}
    \end{exercice}
    
    
     \begin{exercice}
      Une voiture de course fait un tour de circuit de  14 km en 4 minutes à vitesse constante. Alors \ldots
      \begin{ChoixQCM}{4}
      \item En une heure, elle parcourt 280 km
      \item Elle a parcouru 3,5 km par minute
      \item Elle parcourt en 12 minutes trois fois plus de distance
      \item Elle roule en moyenne à 210 km/h
      \end{ChoixQCM}
\begin{corrige}
     \reponseQCM{bcd}
   \end{corrige}
    \end{exercice}

\end{GroupeQCM}
\end{QCM}

  
