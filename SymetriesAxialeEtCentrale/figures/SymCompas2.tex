
\newcommand{\NodeAngle}[2]{%
    %\pgfextra{
        \pgfmathanglebetweenpoints%
            {\pgfpointanchor{#1}{center}}%
            {\pgfpointanchor{#2}{center}}%
            \global\let\MyAngle\pgfmathresult
    }%}

    % #1 premier point              ---- Distance entre 2 nodes ----
    % #2 second point
    % On récupère le résultat dans \MyDist
\makeatletter
\newcommand{\NodeDist}[2]{%
    \pgfpointdiff{\pgfpointanchor{#1}{center}}
                 {\pgfpointanchor{#2}{center}}
    % no need to use a new dimen
    \pgf@xa=\pgf@x
    \pgf@ya=\pgf@y
    % to convert from pt to cm   
    \pgfmathparse{veclen(\pgf@xa,\pgf@ya)/28.45274}
    \global\let\MyDist\pgfmathresult % we need a global macro   
}
\makeatother

% ######################
%     Dessin crayon    #
% ######################

\newcommand{\Crayon}[1]{%
    \begin{scope}[scale=.7,#1]
    \fill[gray!60] (-.2,4.8) -- (.2,4.8)
                -- (.2,.8) --(.1,.65)
                -- (0,.8) -- (-.1,.66)
                -- (-.2,.8) -- cycle ;
    \draw[color=white] (0,4.8) -- (0,.8 );
    \fill[black] (-.2,4.3) -- (0,4.27)
                -- (.2,4.3) -- (.2,4.8) arc(30:150:0.23cm) ;
    \fill[brown!50] (-.2,.8)
        -- (0,0) node[coordinate,pos=0.75](a){}
        -- (.2,.8) node[coordinate,pos=0.25](b){}
        -- (.1,.65) -- (0,.8) -- (-.1,.66) -- cycle;
    \fill[gray] (a) -- (0,0) -- (b) -- cycle;
    \end{scope}
}



% ######################
%   Dessin du compas   #
% ######################

\NewDocumentCommand{\Compas}{smm}{%

    \IfBooleanTF{#1}{%
    % with *
    % keep distance between extemities
    }{%
    % without *
    % calulation of the distance between extemities
    \NodeDist{#2}{#3}
    }

    \NodeAngle{#2}{#3}

    \def\L{6} % taille des branches du compas

    % calcul de l'angle de l'ouverture
    \pgfmathsetmacro{\AngleCP}{asin(\MyDist/(2*\L))}

    \begin{scope}[shift=(#2)]
    \begin{scope}[%
        join=round,
        rotate=\MyAngle,
        shift=(270-\AngleCP:-\L)
        ]

    % branche pointe sèche
    \draw[rotate=-\AngleCP,fill=gray!80]
        (0,0)--(0,-\L)--(-.2,-\L+.8)--(-.2,0)--cycle ;
    \draw[rotate=-\AngleCP,fill=gray!05]
        (0,-\L+.8)--(0,-\L)--(-.2,-\L+.8)--cycle ;

    % branche crayon
    \draw[rotate=\AngleCP,fill=gray!80]
        (0,0)--(0,-\L)--(.2,-\L+.8)--(.2,0)--cycle ;

    \begin{scope}[rotate=\AngleCP,shift={(0,-\L)}]
    \Crayon{rotate=-12}
    \draw[fill=gray!25] (\L/30,\L/5) circle (\L/36) ;
    \fill[gray!5] (\L/30,\L/5)
            -- ++(30:\L/36) arc (30:45:\L/36) -- cycle ;
    \fill[gray!5] (\L/30,\L/5)
            -- ++(210:\L/36) arc (210:225:\L/36) ;
    \filldraw (\L/30,\L/5) circle (.02) ;
    \end{scope}

    % haut du compas
    \draw[fill=gray!80] (-.1,0) rectangle (.1,.7) ;
    \draw[fill=gray!25] (0,0) circle (.25) ;
    \fill[gray!5] (0,0) -- (30:.25) arc (30:45:.25) -- cycle ;
    \fill[gray!5,rotate=180] (0,0) -- (30:.25) arc (30:45:.25) -- cycle ;
    \filldraw (0,0) circle (.05) ;
    \end{scope}
    \end{scope}
}





\scalebox{.2}{
\begin{tikzpicture}	
  
    
\def \CharSize {3};
\def \BulletSize {2};

\coordinate (A) at (0,0) ;
\coordinate (P) at (2,4) ;
\coordinate (M) at (-4,2) ;
\coordinate (N) at (4,-2) ;
\coordinate (S) at (-2,-4) ; %S est le symétrique de P par rapport à d

%Compas
\Compas{N}{S}

%droite d
\draw[very thick] (-8,4)--(8,-4);
\draw (-6,4) node [scale=\CharSize]{$d$};

%point P
\draw (P) node [above,scale=\CharSize]{P};
\draw (P) node[scale=\BulletSize]{$\times$};

%arcs de cercle marqués par le compas
\draw [very thick, color=gray](M) arc(198.4:185:6.32);
\draw [very thick, color=gray](M) arc(198.4:211.8:6.32);
\draw [very thick, color=gray](N) arc(-71.6:-58.6:6.32);
\draw [very thick, color=gray](N) arc(-71.6:-84.6:6.32);

\draw [very thick, color=gray](S) arc(198.4:185:6.32);
\draw [very thick, color=gray](S) arc(198.4:211.8:6.32);
\draw [very thick, color=gray](S) arc(-71.6:-58.6:6.32);
\draw [very thick, color=gray](S) arc(-71.6:-84.6:6.32);

%\draw [color=red](P) circle (6.32);

\end{tikzpicture} 
 } %fin de la scalebox
