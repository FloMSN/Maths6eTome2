%\section{Une section}

% remarque : pour qu'un mot se retrouve dans le lexique : \MotDefinition{asymptote horizontale}{} 

\begin{aconnaitre}
Pour \MotDefinition{alléger l'écriture d'une expression littérale}{}, on peut supprimer le signe de multiplication "$\cdot$" devant une lettre ou une parenthèse.
\end{aconnaitre}

\begin{aconnaitre}
\begin{minipage}[t]{0.38\linewidth}
Pour tout nombre $a$, on peut écrire : 
 \end{minipage} \hfill%
\begin{minipage}[t]{0.58\linewidth}
\textcolor{H1}{$\pmb{a \cdot a = a^2}$} \hspace{1em} (qui se lit  « $a$ au carré ») \\[0.5em]
\textcolor{H1}{$\pmb{a \cdot a \cdot a = a^3}$} \hspace{1em} (qui se lit « $a$ au cube »).
 \end{minipage} \\
\end{aconnaitre}

\begin{methode*1}[Écrire une expression en respectant les conventions]

\begin{remarque}
On ne peut pas supprimer le signe "$\cdot$" entre deux nombres.
\end{remarque}
 
\begin{exemple*1}
Supprime les signes "$\cdot$", lorsque c'est possible, dans l'expression suivante : $A = 5 \cdot x + 7 \cdot (3 \cdot x + 2 \cdot 4)$.
\begin{center}
 \begin{tabular}{lcl}
$A = 5\,\textcolor{H1}{\cdot}\,x + 7\,\textcolor{H1}{\cdot}\,(3\,\textcolor{H1}{\cdot}\,x + 2\,\textcolor{H1}{\cdot}\,4)$ & $\longrightarrow$ & On repère tous les signes "\textcolor{H1}{$\cdot$}" de l'expression. \\ %dark green and bolt
$A = 5x + 7(3x + 2 \cdot 4)$ & $\longrightarrow$ & On supprime les signes "\textcolor{H1}{$\cdot$}" devant une lettre \\ %dark green and bolt
& &  ou une parenthèse.
  \end{tabular}
 \end{center}
\end{exemple*1}

\exercice  
Simplifie les expressions en supprimant le signe "$\cdot$" lorsque c'est possible :\\[1em]
$B = b \cdot a =$ \ldots \ldots;\\[1em]
$C = 5 \cdot x \cdot x \cdot x =$ \ldots \ldots \ldots \ldots;\\[1em]
$D = (3,7 \cdot y - 1,5 \cdot z + 0,4 \cdot 3,5) \cdot 9 =$ \ldots \ldots \ldots \ldots \ldots \ldots \ldots \ldots.
%\correction

\exercice  
Replace les signes "$\cdot$" dans chacune des expressions suivantes :\\[1em]
$E = 12\,a\,c + 35\,a\,b - 40\,b\,c$ \dotfill;\\[1em]
$F = 1,2\,a\,b\,c$ \dotfill;\\[1em]
$G = 5,6\,(x^2 - 2,5\,y + 32)$ \dotfill.
%\correction

\end{methode*1}
 
 %%%%%%%%%%%%%%%%%%%%%%%%%%%%%%%%%%%%%%%%%%%%%%%%%%%%%%%%%%%%%%%%%%%%%%%%
 %%%%%%%%%%%%%%%%%%%%%%%%%%%%%%%%%%%
%%%%%%%%%%%%%%%%%%%%%%%%%%%%%%%%%%%
%MiseEnPage
%%%%%%%%%%%%%%%%%%%%%%%%%%%%%%%%%%%
\newpage
%%%%%%%%%%%%%%%%%%%%%%%%%%%%%%%%%%%
%%%%%%%%%%%%%%%%%%%%%%%%%%%%%%%%%%%
 
 
\begin{aconnaitre}
Pour \MotDefinition{calculer une expression littérale pour une certaine valeur des lettres}{}, il suffit de remplacer les lettres par ces valeurs.
\end{aconnaitre}

\begin{methode*1}[Remplacer des lettres par des nombres]
 
\begin{exemple*1}
Calcule l'expression $A = 5x(x + 2)$ pour $x = 3$ :
\begin{center}
 \begin{tabular}{lcl}
$A = 5 \cdot x \cdot (x + 2)$ & $\longrightarrow$ & On replace les signes "$\cdot$" dans l'expression A. \\
$A = 5 \cdot \textcolor{H1}{\pmb{3}} \cdot (\textcolor{H1}{\pmb{3}} + 2)$ & $\longrightarrow$ & On remplace la lettre $x$ par sa valeur \textcolor{H1}{3}. \\ %dark green and bolt
$A = 15 \cdot 5$ & $\longrightarrow$ &  On effectue les calculs. \\
$A= 75$ & &
  \end{tabular}
 \end{center}
\end{exemple*1}

\exercice  
Calcule les expressions suivantes pour $x = 2$ puis pour $x = 6$ :\\[1em]
$B = 3\,x(x + 5)$ \dotfill;\\

\dotfill \\[1em]
$C = 7x - x^2$ \dotfill;\\

\dotfill \\[1em]
$D = x^3 + 3x^2 - x$\dotfill.\\

\dotfill
%\correction

\exercice  
Calcule les expressions pour $a = 3$ et $b = 5$ :\\[1em]     
$E = 4a + 5b - 56$ \dotfill;\\

\dotfill \\[1em]
$F = a^3 + b^2 + 7ab$ \dotfill;\\

\dotfill \\[1em]
$G = 2(5a + 3b + 1)$ \dotfill;\\.

\dotfill \\[1em]
%\correction

\end{methode*1}
