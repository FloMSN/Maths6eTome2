
\begin{TP}[Boîte noire \ldots]

\partie{Pour bien démarrer \ldots}
\begin{enumerate}
 \item Voici un programme de calcul : \label{CalcLit_TP1}
 \begin{itemize}
  \item \textbf{\textcolor{C2}{Choisir un nombre}} ;
  \item \textbf{\textcolor{C2}{Multiplier ce nombre par 3}} ;
  \item \textbf{\textcolor{C2}{Ajouter 4 au résultat précédent}}.
  \end{itemize}
 Appliquez ce programme pour les nombres : 3 ;  5 et 2,5.
 \item On considère l'expression : $A = 3 x + 4$. Calculez $A$ pour $x = 5$ puis pour $x = 2,5$. Que remarquez-vous ? Expliquez pourquoi. \label{CalcLit_TP2}
 \item Quel programme de calcul correspond à l'expression $B = 7 x - 3$ ?
 \item Essayez de construire un programme de calcul permettant d'obtenir 5 quand on choisit 2 pour nombre de départ. Y a-t-il une seule solution selon vous ?
 \item Achille a écrit un programme de calcul sur son cahier mais il l'a oublié chez lui. Il avait noté sur une feuille à part le tableau suivant : \label{CalcLit_TP3}
 \vspace{0.3cm}
 \begin{center}
  \renewcommand*\tabularxcolumn[1]{>{\centering\arraybackslash}m{#1}}
  \begin{Ctableau}{0.8\linewidth}{4}{c}
   \hline
Nombre de départ & 2 & 4 & 17 \\\hline
Résultat du programme & 9 & 11 & 24 \\\hline
   \end{Ctableau}
 \end{center}
 \vspace{0.3cm}
 À partir de ce tableau, pouvez-vous retrouver un programme de calcul qui conviendrait ?
 \item À l'aide de ce programme, recopiez le tableau précédent puis complétez-le avec trois nouveaux nombres de départ : 5,5 ; 7 et 3,1.
 \item Donnez l'expression avec la lettre $x$ qui correspond à ce programme.
 \item Voici un autre tableau de valeurs :
 \vspace{0.3cm}
 \begin{center}
  \renewcommand*\tabularxcolumn[1]{>{\centering\arraybackslash}m{#1}}
  \begin{Ctableau}{0.8\linewidth}{4}{c}
   \hline
Nombre de départ & 2 & 10 & 1,5 \\\hline
Résultat du programme & 5 & 21 & 4 \\\hline
   \end{Ctableau}
 \end{center}
 \vspace{0.3cm}
Leïla dit que l'expression $C = 3 x - 1$ pourrait parfaitement convenir à un tel tableau. Expliquez pourquoi elle se trompe.
 \item Trouvez un programme de calcul et l'expression associée qui conviendrait pour ce nouveau tableau.
 \end{enumerate}
        
\partie{Boîte noire}

Quand on rentre un nombre dans une boîte noire, elle exécute un programme de calcul pour fournir un résultat. \\[0.5em]
L'objectif de cette partie est de construire des boîtes noires puis d'essayer de démasquer les boîtes noires d'un autre groupe. \\[0.5em]
\begin{enumerate}
\setcounter{enumi}{9}
 \item Vous allez construire deux boîtes noires : une facile et une difficile. La construction de ces boîtes doit rester secrète pour garder le mystère. Pour chacune de ces deux boîtes, il faut :
 \vspace{0.3cm}
 \begin{itemize}
  \item Trouver un programme de calcul, comme à la question \ref{CalcLit_TP1} (les nombres utilisés doivent être des entiers plus petits que 10) ;
  \item Trouver l'expression qui correspond, comme à la question \ref{CalcLit_TP2} ; 
  \item Faire un tableau comme à la question \ref{CalcLit_TP3}, avec trois valeurs et les résultats obtenus.
  \end{itemize}
 \vspace{0.3cm}
Pour la boîte facile, le programme ne peut comporter qu'une seule fois la lettre $x$. \\[0.5em]
Pour la boîte difficile, le programme ne peut comporter qu'un seul terme avec  $x^2$.
 \vspace{0.3cm}
 \item Une fois que vous avez construit vos boîtes, écrivez les deux tableaux de valeurs sur une même feuille. Vérifiez bien que vos tableaux sont corrects ! Échangez cette feuille avec la feuille d'un autre groupe.
 \item Quand un groupe pense avoir réussi à décoder une boîte noire, il peut s'en assurer en demandant au groupe qui l'a créée le résultat que donnerait la boîte noire pour la valeur de leur choix. Le défi est relevé quand un groupe est capable d'écrire sur une feuille le programme et l'expression correspondante pour chacune des boîtes noires.
 \end{enumerate}
\underline{Attention} : Si un groupe s'est trompé dans ses calculs pour réaliser le tableau alors c'est ce groupe qui aura perdu le défi !

\end{TP}

