%\section{Une section}

% remarque : pour qu'un mot se retrouve dans le lexique : \MotDefinition{asymptote horizontale}{} 

\begin{methode*1}[Additionner deux nombres relatifs]

\begin{aconnaitre}
Pour \MotDefinition{additionner deux nombres relatifs de même signe}{}, on additionne leurs valeurs absolues et on garde le signe commun.

Pour \MotDefinition{additionner deux nombres relatifs de signes contraires}{}, on soustrait leurs valeurs absolues et on prend le signe de celui qui a la plus grande distance à zéro.
\end{aconnaitre}

\begin{exemple*1}
Effectue l'addition suivante : $A = (-2) + (-3)$.
\begin{tabular}{ll} 
$A = (-2) + (-3)$ & $\rightarrow$ On veut additionner deux nombres négatifs. \\
$A = -(2 + 3)$ & $\rightarrow$ On additionne les valeurs absolues \\
 & \phantom{$\rightarrow$} et on garde le signe commun : $-$. \\
$A = -5$ & $\rightarrow$ On calcule.
\end{tabular}
 \end{exemple*1}
 
 \begin{exemple*1}
Effectue l'addition suivante : $B = (-5) + (+7)$.
\begin{tabular}{ll} 
$B = (-5) + (+7)$ & $\rightarrow$ On veut additionner deux nombres de signes différents. \\
$B = +(7 - 5)$ & $\rightarrow$ On soustrait leurs valeurs absolues et on écrit \\
 & \phantom{$\rightarrow$} le signe du nombre qui a la plus grande valeur absolue. \\
$B = +2$ & $\rightarrow$ On calcule.
\end{tabular}
 \end{exemple*1}
 
 \exercice  
Effectue les additions suivantes : \\[.1em]
\begin{colenumerate}{1}
 \item $(+7) + (+4 )=$ \dotfill; \\[.1em]
 \item $(+12) + (-15)=$ \dotfill; \\[.1em]
 \item $(-7) + (+19)=$ \dotfill; \\[.1em]
 \item $(-11) + (-9)=$ \dotfill; \\[.1em]
 \item $(+1) + (+3) + (-2)=$ \dotfill; \\[.1em]
 \item $(-2) + (-6) + (+7)=$ \dotfill; \\[.1em]
 \item $(-10,8) + (+2,5)=$ \dotfill; \\[.1em]
 \item $(+25,2) + (-15,3)=$ \dotfill; \\[.1em]
 \item $(-21,15) + (+21,15)=$ \dotfill.
 \end{colenumerate}
%\correction

 \end{methode*1}

%%%%%%%%%%%%%%%%%%%%%%%%%%%%%%%%%%%%%%%%%%%%%%%%%%%%%%%%%%%%%%%%%%

\begin{methode*1}[Soustraire deux nombres relatifs]

\begin{aconnaitre}
\MotDefinition{Soustraire un nombre relatif}{} revient à additionner son opposé.
\end{aconnaitre}

 \begin{exemple*1}
Effectue la soustraction suivante : $C = (-2) - (-3)$.
\begin{tabular}{ll} 
$C = (-2) - (-3)$ & $\rightarrow$ On veut soustraire le nombre $-3$. \\
$C = (-2) + (+3)$ & $\rightarrow$ On additionne l'opposé de $-3$. \\
$C = + (3 - 2)$ & $\rightarrow$ On additionne deux nombres de signes différents donc \\
 & \phantom{$\rightarrow$} on soustrait leurs valeurs absolues et on écrit \\
 & \phantom{$\rightarrow$} le signe du nombre qui a la plus grande valeur absolue. \\
$C = +1$ & $\rightarrow$ On calcule. \\
\end{tabular}
 \end{exemple*1}

\exercice
Transforme les soustractions en additions et effectue :\\[.5em]
\begin{colenumerate}{1}
 \item $(+5) - (-6)=$ \dotfill; \\[.1em]
 \item $(-3) - (+2)=$ \dotfill; \\[.1em]
 \item $(+4) - (+8)=$ \dotfill; \\[.1em]
 \item $(-7) - (-3,8)=$ \dotfill; \\[.1em]
 \item $(-2,3) - (+7)=$ \dotfill; \\[.1em]
 \item $(+6,1) - (-2)=$\dotfill.
 \end{colenumerate}
%\correction

\exercice
Effectue les soustractions suivantes : \\[.1em]
\begin{colenumerate}{1}
 \item $(+3) - (-6)=$ \dotfill; \\[.1em]
 \item $(-3) - (-3)=$ \dotfill; \\[.1em]
 \item $(+7) - (+3)=$ \dotfill; \\[.1em]
 \item $(-5) - (+12)=$ \dotfill; \\[.1em]
 \item $(+2,1) - (+4)=$ \dotfill; \\[.1em]
 \item $(-7) - (+8,25)=$\dotfill.
 \end{colenumerate}
%\correction

 \end{methode*1}
 
 %%%%%%%%%%%%%%%%%%%%%%%%%%%%%%%%%%%%%%%%%%%%%%%%%%%%%%%%%%%%%%%%%%

\begin{methode*1}[Simplifier l'écriture d'un calcul]

\begin{aconnaitre}
Dans une suite d'additions de nombres relatifs, on peut supprimer les signes d'additions et les parenthèses autour d'un nombre.

Un nombre positif écrit en début de calcul peut s'écrire sans son signe.
\end{aconnaitre}

\begin{remarque}
Dans le cas d'une expression avec des soustractions, on peut se ramener à une suite d'additions.
 \end{remarque}

 \begin{exemple*1}
Simplifie l'expression $D = (+4) + (-11) - (+3)$ :
\begin{tabular}{ll} 
$D = (+4) + (-11) + (-3)$ & $\rightarrow$ On transforme les soustractions en additions \\
 & \phantom{$\rightarrow$} des opposés. \\
$D = +4 - 11 - 3$ & $\rightarrow$ On supprime les signes d'additions et \\
 & \phantom{$\rightarrow$} les parenthèses autour des nombres. \\
$D = 4 - 11 - 3$ & $\rightarrow$ On supprime le signe $+$ en début de calcul. \\
\end{tabular}
 \end{exemple*1}

\exercice
Simplifie les écritures suivantes :
\vspace{0.8em}
\begin{enumerate}
 \item $(-5) - (-135) + (+3,41) + (-2,65)$ \dotfill \\[.4em]
 
 \dotfill; \\[.4em]
 
 \item $(+18) - (+15) + (+6) - (-17)$ \dotfill \\[.4em]
 
 \dotfill.
 \end{enumerate}
%\correction

\end{methode*1}
